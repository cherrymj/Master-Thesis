%!TEX root = Thesis.tex
\chapter*{Zusammenfassung}
\label{cha:zusammenfassung}

Parken ist zu einem der größten Probleme in Großstädten geworden. Man verliert
viel Zeit, Brennstoff und Geld beim Versuch einen freien Parkplatz zu finden.
Studien zeigen, dass die Parkplatzsuche für mehr als 10\% des
Verkehrsaufkommens in Städten verantwortlich ist und es bis zu 20 Minuten
dauern kann, bis man einen Parkplatz gefunden hat.

Daher beschäftigt sich diese Massenarbeit mit der Vorhersage der Belegung von
Parkplätzen. Diese Informationen können die Parkplatzsuche erheblich
erleichtern, indem man diese in Planungsmethoden integriert, die häufig in
Handys und GPS Navigationssystemen schon vorhanden sind. Wir präsentieren
einen automatisierten Ansatz der Sammlung und Interpretation der
Belegungsdaten von Parkplätzen mit Hilfe einer mobilen Platform. Dieser Ansatz
erlaubt die Routenplanung mit der Hilfe von nicht nur räumlichen
Informationen, sondern auch der Belegungsinformation über die Parkplätze.

Der vorgeschlagene Ansatz verwendet die Kamera des Fahrzeuges, um die
geparkten Autos zu erkennen und diese den Parkplätzen zuzuweisen. Darüber
hinaus wird die Belegungswahrscheinlichkeit jedes Parkplatzes über all Läufe
hinweg geschätzt. Weiterhin lässt sich der Ansatz leicht erweitern, um die
Belegungsinformationen an bestimmten Tagen oder zu bestimmten Uhrzeiten zu
ermitteln.

Außerdem stellen wir auch einen Planer vor, der nicht nur die Position,
sondern auch die Belerungswahrscheinlichkeiten mit in Betracht zieht. Der
Planner sucht nach dem optimalen Weg, welcher die Zeit, die ist nötig um einen
freien Platz zu finden und anschließend ein Ziel zu Fuß zu erreichen,
minimiert.

Wir haben das System an Hand von echten Daten, die mit einem mobilen Roboter
an mehreren Tagen auf einem Parkplatz gesammelt wurden, ausgewertet.
