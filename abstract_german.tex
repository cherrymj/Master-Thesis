%!TEX root = Thesis.tex
\chapter*{Zusammenfassung}
\label{cha:zusammenfassung}

Das Parken ist eine die größte Probleme der großen Städte geworden. Die Leute
verlieren Zeit, Brennstoff und Geld in den Versuch eine freie Parkplatz zu
finden. Die Studien zeigen, dass die Menschen in die Suche nach einem
Parkplatz mehr als 10\% von den Stadtverkehrsstaus erstellen können, und die
Zeit, bis sie einen Parkplatz finden, kann im Ergebnis bis zu 20 Minuten
erreichen.

Deswegen es ist nützlich die Belegung von Parkräume vorherzusagen und in die
Karte einzuzeichnen. Diese Information kann die Sucherfahrung nach einem
freien Parkplatz dramatisch verbessern, wenn man die in die Plannenmethoden
integriert. Die Plannenmethoden sind often in Handys und GPS
Navigationssystemen anwesend.

Wir präsentieren einen automatisierten Ansatz der Sammlung und Interpretation
der Belegungsinformation von Parkplätze mit Hilfe der Mobileplatform. Das
erlaubt eine Routeplanung mit der Hilfe nicht nur von Räumlicheinformation,
sondern auch der Belegunginformation über die Parkplätze.

Der vorgeschlagene Ansatz nutzt eine Fahrzeug Kamera-Setup um die geparkte
Autos, die sich im Region von Interesse befinden, rückfällig zu ermitteln und
in die Karte einzuzeichnen. Die Kamera-Setup ist auch für eine Bewertung der
Belegungswahrscheinlichkeit von jeder Parkposition benuzt. Der
Strukturplan(framework) ist leicht erweiterbar um die
Belegungsinformationanfragen für bestimmtes Datum und sogar Teilen des Tages
zu berücksichtigen.

(Futhermore) Wir stellen auch den Planner, der sich auf die
Belegungwahrscheinlichkeit von jeden einzelnen gefundenen Parkplatz verlässt,
vor. Der Planner sucht nach dem optimalen Weg unter dem Kriterion, dass die
Zeit, die ist nötig um einen freien Platz zu finden und um eine Destination
zum Fuß zu erreichen, minimisiert sein muss.
