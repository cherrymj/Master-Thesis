%!TEX root = Thesis.tex
\chapter{Background}\label{cha:background}
In order to be able to solve the given problem we will have to rely on several
well known approaches, which we will briefly introduce in this chapter.
\section{Occupancy Grids}\label{sec:occupancy_grids}

In this thesis we partly make use of the grid maps to model the
environment with respect to modeling the probability of cars to be
present in the particular part of the world. Grid maps partition the
space into a grid of rectangular cells. Each grid contains information
about the corresponding area in the environment. In this thesis we
make use of one particular realization of grid maps --- the occupancy
grid maps. Occupancy grid maps assume that each grid cell is either
occupied by an obstacle or free. In our case, each cell therefore
stores the probability that the particular cell is occupied by a car.
The occupancy grid maps are an efficient approach for representing
uncertainty. Grid maps allow for detailed representation of an
environment without a predefined feature extractor. As they have the
ability to represent occupied space, empty space and unknown
 (unobserved) space, grid maps are well suited for tasks such as path-
planning, obstacle avoidance and exploration. In contrast to most
representations based on features, grid maps offer constant time
access to cells. However, grid maps suffer from discretization errors.

\section{Object Detection}\label{sec:object_detection}

In this thesis we make extensive use of
the object detection to be able to detect parked cars from visual information.
Based of the work of~\cite{dalal2005} we make use of the histograms of
oriented gradients (HOG) features placed in a dense grid on the query images.
The histograms from the training data, are  divided into positive and negative
groups via support vector machines (SVM), presented by~\cite{svm2003}, as
shown in Figure~\ref{fig:det_to_svm}.

Let us shortly describe how the HOG descriptor is formed. The image is
partitioned into $8 \times 8$ pixel blocks. In each block we compute the
histogram of the gradient orientations. This histogram, after the
normalization is invariant to changes in light, small deformations etc.

The HOG/SIFT representation has several advantages. It captures edge or
gradient structure that is very characteristic of local shape, and it does
so in a local representation with an easily controllable degree of
invariance to local geometric and photometric transformations:
translations or rotations make little difference if they are much smaller
than the local spatial or orientation bin size.


\section{Markov Decision Processes}
\label{sec:markov_descision_processes}
Markov decision processes are used to carry out complex decisions in a fully
observable environment with a stochastic transition model. We make use of the
MDPs to find an optimal way of searching for a free parking lot.  A
specification of the outcome probabilities for each action in each possible
state is called a transition model, we will use $T(s \mid s', a)$ to denote
the probability of reaching state $s'$ if and action $a$ is carried out in
state $s$. The transitions in the model are assumed to be Markovian. That is,
the probability of reaching $s'$ from $s$ depends only on $s$ and not on the
history of earlier states. We also need to define the reward function $R(s, a,
s')$, that defines a reward for reaching state $s'$ from state $s$ carrying
out action $a$. Overall, the MDP is defined by three components:

\begin{itemize}
    \item Initial state: $S_0$
    \item Transition Model: $T(s \mid s', a)$
    \item Reward Function: $R(s, a, s')$
\end{itemize}

These components allow us to define the utility function, which is a measure
of how good the state is in the long run:

\begin{equation}
U^{\pi}(s) = E\left[\sum_{t=0}^{\infty} \gamma^t R(s_t) \mid \pi,s_0 = s \right]
\end{equation}

The optimal action is then chosen using the Maximum Utility Principle, that
is, choose: the action that maximizes the expected utility of the subsequent
state:

\begin{equation}
\label{lbl:optimal_policy}
\pi^*(s) = \argmax_a \sum_{s'} T(s \mid s', a)U(s')
\end{equation}

Provided, that the utility of the state is the expected sum of discounted
rewards from that point onwards, we can define the utility via Bellman
Equation:
\begin{equation}
\label{lbl:bellman_equation}
U(s) = R(s) + \gamma \max_a \sum_{s'}T(s \mid s', a)U(s')
\end{equation}

In order to solve the MDP there are two common strategies to be used: variable
iteration and policy iteration. We will only focus on the second, which is
briefly introduced here. The policy iteration algorithm alternates the
following two steps, beginning from some initial policy $\pi_0$,

\begin{itemize}
    \item \emph{Policy evaluation:} given a policy $\pi_i$,
    calculate $U_i = U^{\pi_i}$, the utility of each state if $\pi_i$, were to be
    executed.
    \item \emph{Policy improvement:} Calculate a new MEU policy
    $\pi_{i+l}$, using one-step look- ahead based on $U_i$ (as in
    Equation~\ref{lbl:optimal_policy}).
\end{itemize}

The algorithm terminates when the policy improvement step yields no change in
the utilities. The given method can be shown to converge in $O(n^3)$ where $n$
is the number of vertices in the graph.

%chapter background (end)



