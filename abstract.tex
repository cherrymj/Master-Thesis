%!TEX root = Thesis.tex
\chapter*{Abstract}
\label{cha:abstract}

Parking has become one of the biggest problems of the big cities. People lose
time, fuel and money in an attempt to find a free parking spot. The studies
show, that the people ``cruising'' in the search for a parking space generate
up to 15\% of an overall city traffic and the times until they eventually find
a parking space can reach 20 minutes.

Therefore it comes in handy to map and predict the occupancy of parking
spaces. This information, when integrated into planners commonly found in cell
phones and GPS navigation systems, can drastically improve the experience of
searching for an empty parking lot.

We present an automated approach of gathering and interpreting the parking lot
occupancy information using a mobile platform.

It allows for planning the route not only with respect to spatial information
but also taking into account parking lots' occupancy information.

The proposed approach uses an in-vehicle camera setup to repeatedly detect and
map the parked cars in the area of interest and to estimate the occupancy
probability of each found position suitable for parking throughout all runs.
The framework is easily expendable to account for querying for occupancy
information on specific date or even time of day.

Furthermore, we also introduce the planner that relies on the occupancy
probability of each found parking lot to find the optimal path in the means of
time spent on searching for an unoccupied parking place and walking from the
found one to the destination by foot.



