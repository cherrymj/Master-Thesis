%!TEX root = Thesis.tex
\chapter*{Abstract}
\label{cha:abstract}

Parking has become one of the biggest problems of the big cities. People lose
time, fuel and money when searching for a free parking spot. Studies show,
that the people cruising in the search for a parking space can generate more
than 10\% of an overall city traffic and the times until they eventually find
a parking space can reach 20 minutes.

Therefore we investigate the problem of how to map and predict the occupancy
of parking spaces. This information, when integrated into navigation devices
as they are today found in cell phones and GPS navigation systems, has the
potential to drastically improve the experience of searching for an empty
parking lot.

We present a proof of concept of an automated approach for gathering and
interpreting the parking lot occupancy information using a mobile platform. It
allows for planning the route not only with respect to spatial information but
additionally taking into account parking lots' occupancy information.

The proposed approach uses an in-vehicle camera setup to repeatedly detect the
parked cars and assign them to the parking lots in the area of interest. It
estimates the occupancy probability of each found position suitable for
parking throughout all runs. The framework is easily expandable to account for
querying for occupancy information on specific date or even time of day.

Furthermore, we introduce a planner that relies not only on the position but
also on the occupancy probability of each parking lot in order to find the
path that minimizes the expected time spent on searching for an unoccupied
parking place and walking from the found one to the destination by foot.

We evaluated our system based on real world data gathered with a mobile robot
over several days in a real parking lot.



