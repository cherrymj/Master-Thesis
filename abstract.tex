\chapter{Abstract}
\label{cha:abstract}

As there are more and more cars on the streets from day to day it gets more and
more complicated to find a parking space, especialy in the center of the city,
where one has spacial constraints on building new parking lots. 

Therefore it comes in handy to map and predict the occupancy of parking spaces.
This information, when integrated into planners commonly found in cell phones
and GPS navigation systems, can drastically improve the experience of searching
for a parking lot.

In this thesis we present an automated approach of gathering and interpreting
the parking lot occupancy information using a mobile platform. 

It allows for planning the route not only
with respect to spatial information but also taking into account parking lot
occupancy information.

The proposed approach repeatedly detects and maps the parked cars in the area of interest
and estimates the occupancy probability of each found position throuout all runs. The framework
is easily extendable to account for searching for occupancy on specific date or
even time of day.

Furthemore, we also introduce the planner that relies on the occupancy
probability of each found parking lot to find the optimal path in the means of time spent on
searching for a parking lot, while considering spacial constraints like the
distance to the end destimation.



