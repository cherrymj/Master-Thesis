%!TEX root = Thesis.tex

\chapter{Conclusion and Further Work} % (fold)
\label{cha:conclusion_and_further_work}

    The task of this thesis was to develop a framework for supporting drivers
    in finding free parking spaces. The system to solve this task should
    perceive cars in the environment, model parking places, and suggest
    trajectories towards free parking lots.

    As the result of this thesis we present a proof of concept of an
    integrated system, that only relies on on-board perception, models the
    parking situation and generates optimal trajectories for finding a free
    parking space and reaching the target location as fast as possible. Our
    system automatically quantifies the occupancy probability of each parking
    lot in an area of interest and provides a behavior that minimizes the
    expected time for reaching the target destination.

    Our approach requires only on-board sensing to perceive the environment.
    So far, we used a camera and a laser range finder to observe the scene and
    to detect cars from the perceptual input. This sub-task was solved by a
    supervised learning approach using a support vector machine and histograms
    of oriented gradients (HOG) as features. The detections are fused into a
    model of the environment that allows for building up a representation of
    the scene depicting the occupancy probability of each parking lot. Based
    on this representation, we define a Markov Decision Process that allows us
    to generate the optimal path, given the current knowledge, to find a
    parking lot and at the same time park as close as possible to the desired
    target location.

    We present a series of experiments carried out with the use of the Obelix
    robot at the campus of the University of Freiburg. We evaluate HOG based
    visual detection of the cars, assignment of the detections to the parking
    lots, estimation of their occupancy probability and planning with the use
    of MDPs. The visual detection fused with the laser measurements shows
    steady car detection rates, where approximately 90\% of the parking lots
    are labeled correctly. The MDP based planner was evaluated on the real
    occupancy data acquired from the parking lot at the University Freiburg
    campus. It is shown to find the optimal route to the goal, while adopting
    to current system parameters, such as walking and driving speeds.

    There is, however, still place for the future work. The visual detection
    rates can be improved by adjusting the parameters of the HOG descriptors
    and adjusting the training datasets. This, however, requires extensive
    additional study.

    Also the component integration can be improved to form an more consistent
    framework. In the current work the planner is a stand-alone application,
    written in Octave. Though it was not in the scope of this thesis, it may
    be subject to the future work to integrate it with the robot's planner
    system.

    The improvements can also be made to the parking lots modeling. Our
    current approach relies on the pre-defined parking lot positions. It is
    tempting to improve the performance of the occupancy grid maps in order to
    map the parking lots in an autonomous fashion based on the observations of
    the parked cars at different times. In our work we rely on the
    Simultaneous Localization and Mapping (SLAM) approach in order to
    precisely estimate the position of the robot while taking measurements.
    This part is crucial for data association. It would, however, be a great
    improvement to find an way of estimating the parking lots' occupancy
    information relying only on GPS measurements.

    Overall, the system, presented in this thesis as a proof of concept, when
    merged with all further developments, might become a helpful tool in the
    task of parking in densely occupied parking areas. The described system
    may also be used in autonomous vehicles as it provides a way to deal with
    searching for a parking lot in a fully autonomous fashion.

% chapter conclusion_and_further_work (end)
