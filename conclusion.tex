%!TEX root = Thesis.tex

\chapter{Conclusion and Further Work} % (fold)
\label{cha:conclusion_and_further_work}

    The task of this thesis was to develop a proof of concept framework for
    supporting drivers in finding free parking spaces. The system should solve
    this task by perceiving cars in the environment, modeling parking places,
    and suggesting trajectories towards free parking lots.

    As the result of this task we present a proof of concept of an integrated
    system, that only relies on on-board perception, models the parking
    situation and generates optimal trajectories for finding a free parking
    space and reaching the target location as fast as possible. Our system
    automatically quantifies the occupancy probability of each parking lot in
    an area of interest and provides a behavior that minimizes the expected
    time for reaching the target destination.

    Our approach requires only on-board sensing to perceive the environment.
    So far, we use a camera and a laser range finder to observe the scene and
    to detect cars from the perceptual input. We solve this subtask by a
    supervised learning approach using a support vector machine and histograms
    of oriented gradients (HOG) as features. The detections are fused into a
    model of the environment that allows for building up a representation of
    the scene depicting the occupancy probability of each parking lot. Based
    on this representation, we define a Markov Decision Process that allows us
    to generate the optimal path, given the current knowledge, to find a
    parking lot and at the same time park as close as possible to the desired
    target location.

    We present a series of experiments carried out with the Obelix robot at
    the campus of the University of Freiburg. We evaluate the HOG based visual
    detection of the cars, the assignment of the detections to the parking
    lots, and the trajectory planning using MDPs. The visual detection fused
    with the laser measurements shows steady car detection rates, where
    approximately 90\% of the parking lots are labeled correctly. The MDP
    based planner was evaluated on the real occupancy data acquired from the
    parking lot at the University Freiburg campus. It is shown to find the
    optimal route to the goal, while adopting to current system parameters,
    such as walking and driving speeds.

    Although the system solves the given task of this thesis, there is space
    for the future work. The visual detection rates can be improved by
    adjusting the parameters of the HOG descriptors and adjusting the training
    datasets. This, however, requires extensive additional study.

    Whereas the perception and modeling part was written in C++ and runs on
    the robot, the planner is a stand-alone application, written in Octave.
    Thus, for an integrated system, all components need to be implemented on
    the same platform. Though it was not an explicit task of this thesis, it
    may be subject to the future work to integrate it with the robot's planner
    system.

    The improvements can also be made to the parking lots modeling. Our
    current approach relies on the pre-defined parking lot positions as this
    gave the best performance in our settings. It is tempting to improve the
    performance of the occupancy grid maps in order to map the parking lots in
    an autonomous fashion based on the observations of the parked cars at
    different times. In our work we rely on a simultaneous localization and
    mapping approach in order to precisely estimate the position of the robot
    while taking measurements. This part is crucial for data association. It
    would, however, be an interesting improvement to find a way of estimating
    the parking lots' occupancy information relying only on GPS measurements.

    Overall, the system, presented in this thesis is a proof of concept
    implementation. The described system, however, has the potential to be
    used in autonomous vehicles as it provides a way to deal with searching
    for a parking lot in a fully autonomous fashion.

% chapter conclusion_and_further_work (end)
