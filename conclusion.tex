%!TEX root = Thesis.tex

\chapter{Conclusion and Further Work} % (fold)
\label{cha:conclusion_and_further_work}

    We have presented and implemented a proof of concept system that
    integrates perception, mapping and planning for solving a task of
    efficient search for a free on-street parking lot. Our relies on visual
    detection of the cars in the streets.

    Throughout the course of the work we experiment with different methods for
    combining metric information with visual detections such as utilizing
    depth from a stereo camera or a laser range finder. Currently the laser
    range scanner s proven to provide a better result, however it is of great
    interest to build a system that would provide the same level of detail
    adopting purely visual information.

    It is important to have a reliable depth estimate and of the same
    importance is having likewise a good estimate of the position in the
    world. We use SLAM
    (\cite{stachniss11isrr,kuemmerle11auro,kretzschmar10ki}) for estimating a
    precise position of the agent in the world.

    Not only we experiment with perception, but also with mapping. We
    implement the occupancy grids based approach, which, however, suffers from
    the discretization errors as well as from ad-hoc estimation of the
    position (including orientation) of the detected cars in the occupancy
    grid. Eventually we promote the system where we provide additional
    knowledge about where the parking lots are situated. This information has
    to be obtained only once for any new environment.

    This system then provides a good estimate of the occupancy information
    that we estimate via multiple observations of the same spot on different
    days or simply in different times. It may be a subject for future work to
    estimate the occupancy information for particular day of week or
    particular time of a working day, etc. The occupancy information is
    basically a probability of a parking lot to be free when observed.

    This occupancy information along with the position of the parking lots in
    the world allows us to introduce a planner. This planner, unlike greedy A*
    and alike, searches not for the best parking lot in the means of position
    or occupancy, but takes into account the trade-off between these two
    quantities, minimizing the expected time required by an agent to park a
    car and get to the final goal by foot. It is still possible to experience
    difficulties finding the best parking lot, but our approach eliminates the
    high uncertainty of this process and provides an opportunity to carry out
    the parking process in a fully-autonomous fashion, while guaranteeing to
    find the best action in the means of expected time needed for the whole
    process.

    There is however place for future work. To see a fully integrated and
    easily accessible system we will further focus on detecting the parked car
    position purely based on visual data (via improving stereo-camera depth
    acquisition). The other vector of interest is improving the approach for
    mapping in order to leave pre-defined parking lots' positions behind and
    to be able to efficiently map any given environment without any prior
    knowledge about it.

% chapter conclusion_and_further_work (end)
