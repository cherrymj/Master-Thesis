%!TEX root = Thesis.tex
\chapter{Related Works}\label{cha:related_works}

The problem of finding a free parking lot receives more attention with the
never-ending growth of the cities and their population. However, most of the
work is oriented on providing people with simple information on the occupancy
status of a particular parking garage or parking lot.

There is quite significant amount of work done in the field of detection of
the parked vehicles using either a static monocular camera or a set of such
cameras. In the work of~\cite{qizhang06} the authors present an unsupervised
system to monitor the occupied parking slots. They were using a stationary
monocular camera to detect parked vehicles. In their work car detection is
solely based on color, SVM is used to distinguish between occupied and free
parking lots. The authors pre-process the ground truth images of the parking
lots and search for color differences between actual measurements and the
ground truth measured for an empty parking lot.

Another work by~\cite{nicolastrue} is showing a similar approach. Alike with
the previous paper, an unsupervised system for parking lots detection is
presented. The author is using an overhead static camera. He uses human-
labeled parking lots' positions to define the spacial arrangement of the
parking lots. The author distributes chrominance channels of the images from
the camera into bins, storing a histogram of these values for each parking
lot. The histograms are then classified into free vs. occupied using either
k-nearest neighbors or SVM. He also presents a variant of the work based on
classifying color patches that are situated around corner detector's regions
of interest, this however proves to be not sufficiently efficient.

The authors of the next paper ---~\cite{yusnita12} proposed a system to detect
occupancy status of the parking lots of indoor parkings. They are using an
overhead, strictly vertically oriented camera. The parking lots are manually
labeled by a marker that states their occupancy state. For each query parking
lot the authors produce a binary image which is then analyzed to find if the
parking lot is empty or occupied. This process is carried out by comparing the
shape of the detection with the shape of the prior that reassembles the circle
shape of the marker drawn on the floor of the parking slot.

Another work that utilizes a static monocular camera~\cite{tschentscher} shows a
comparison of several different features for occupancy analysis. As in the works
presented above the authors of this paper have an overhead camera pointing to
the parking lots, which are manually labeled in the images. They compare a
couple of visual features such as color histograms, gradient histograms,
difference of Gaussian histograms and Haar features. They also explore different
methods for training a classifier such as k-nearest neighbors, linear
discriminant analysis and SVM. They achieved detection rate of 98\% while
performing in real time.

In the works of~\cite{chingchun10},~\cite{chingjao10}, the authors are using a
static monocular overhead camera and a 3-layer Bayesian hierarchical framework
 (BHF) the authors of the paper specifically addressed the challenges of vacant
parking space inference that come from dramatic luminance variations, shadow
effect, perspective distortion, and the inter-occlusion among vehicles showing
great detection rates.

The authors of this paper~\cite{ichihashi} have developed a system that uses a
system of overhead cameras for occupancy detection in indoor and outdoor
scenes. The focus is put on the clustering algorithms, showing that the
performance of the detector based on the fuzzy c-means (FCM) clustering and
the hyper-parameter tuning by particle swarm optimization (PSO) significantly
outperforms SVM both in speed and accuracy of detection.

However, there is also work done in the field of the in-vehicle detection of
the parking lots. Some works present approaches that rely on the fusion of
sonar/laser data with visual information.

The authors of the next two papers ---~\cite{fintyelvestri} and
\cite{abadvestri} are modeling free parking lots locally based on the 3D sonar
sensor mounted on the car with conjunction with a visual 3D estimation setup
based on the tracking of the interest points in the images seen from different
angles.

In these papers the authors used the ultrasonic sensor and 3D vision sensor
that detects points of interest on the bodies of the parked cars and tracks
them to compare their positions from different points of view. The cars are
then modeled by vertical planes of two different orientations that are fit
into the 3D data from the sensors. Empty parking lots are then modeled by
empty spaces between the planes.

The paper by~\cite{vladimircoric} also moves focus from static overhead
cameras on the developed parking lot to estimating parking lots for on-street
parking. In this paper they use an ultra-sonic sensor mounted to the side of
the car to estimate the parking possibilities along particular streets. The
authors use a straightforward threshold method to distinguish between free and
occupied space. They also make use of an idea that the more time a car was
observed in a place the more it is likely that the space occupied by this car
is a valid parking spot. As a result all measurements were incorporated with
the GPS measurements to form a global map of parked cars, which the authors
used to detect wrongly parked cars.

Another paper, that focuses on the on-vehicle sensor setup for occupancy
modeling by~\cite{schmid11} utilizes an on-board short-range radar. The
measurements from three radars are stored into a 3D occupancy grid, that
represents the local surroundings of the car. Free parking lots are then
detected on the given grid by analyzing occupancy grid cells on curb and on
other parked cars. This provides relatively precise estimation of the parking
lot in 3D that allows for precise parking.

The authors of this paper by~\cite{suhr10} proposed a system that is able to
detect parking lots from 3D data acquired from stereo-based multi-view 3D
reconstruction. The authors select point correspondences using a de-rotation-
based method and mosaic 3D structures by estimating similarity transformation.
While relying solely on this information and not using the odometry they were
able to achieve reliable results with the detection rate of 90\%

Some works, however focus not on car detection, but on detecting the parking
slots markings which can prove to be useful when analyzing a free outdoor
parking.

A fully automatic system for detection of parking slots' markings is presented
by~\cite{suhr13}. The authors are utilizing a tree structure performing a
bottom-up and a top-down approach in order to classify all parking slots as
such and leave out all false detections.
