\chapter{Related Works}
\label{cha:related_works}

The problem of finding a free parking lot receives more attention with the
neverending growth of the cities and their population.
However, most of the work is oriented on providing people with simple
information on the occupancy status of a particular parking garage or parking
lot.
There is quite significant amount of work done in the field of detection of the
parked vehicles using either a static monocular camera or a set of such cameras.
In the works of \cite{qizhang06} the authors present an unsupervised system to
monitor the occupied parking slots. They were using a stationary monocular
camera to detect parked vehicles.
In their work car detection is solely based on color, SVM is used to distinguish
between occupied and free parking lots. The authors preprocess the ground truth
images of the parking lots and search for color differences between actual
measurements and the ground truth measured for an empty parking lot.
\newline 
Another work \cite{nicolastrue} is showing a similar approach. Alike with the
previous paper, an unsupervised system for parking lots detection is presented.
The author is using an overhead static camera. He uses human-labeled parking
lots' positions to define the spacial arrangement of the parking lots. The
author distributes chrominance channels of the images from the camera into bins,
storing a histogram of these values for each parking lot. The histograms are
then classified into free vs. occupied using either k-nearest neighbors or SVM.
He also presents a variant of the work based on classifying color patches that
are situated around corner detector's regions of interest, this however proves
to be not sufficiently efficient. 
\newline
The authors of the next paper \cite{yusnita12} proposed a system to detect
occupancu status of the parking lots of the indoor parkings. They are using an
overhead, strictly vertically oriented camera. The parking lots are manually
labeled by a marker that states their occupancy state. For each query parking
lot the authors produce a binary image which is then analyzed to find if the
parking slot is empty or occupied. This process is carried out by comparing the
shape of the detection with the shape of the prior that reassembles the circle
shape of the marker drawn on the floor of the parking slot.
\newline
Another work that utilizes a static monocular camera \cite{tschentscher} shows a
comparison of several different features for occupancy analysis. As in the works
presented above the authors of this paper have an overhead camera pointing to
the parking lots, which are manually labeled in the images. They compare a
couple of visual features such as color histograms, gradient histograms,
difference of gaussian histograms and Haar features. They also explore different
methods for training a classifier such as k-nearest neighbors, linear
discriminant analysis and SVM. They achieved detection rate of 98\% while
performing in real time.
\newline
In another work, \cite{chingchun10}, \cite{chingjao10}, using a static monocular
overhead camera and a 3-layer Bayesian hierarchical framework (BHF) the authors
of the paper specifically addressed the challenges of vacant parking space
inference that come from dramatic luminance variations, shadow effect,
perspective distortion, and the inter-occlusion among vehicles showing great
detection rates.
\newline
The authors of these papers \cite{fintyelvestri}, \cite{abadvestri} here
modeling free parking lots locally based on the 3D sonar sensor mounted on the
car with conjunction with a visual 3d estimation setup based on the tracking of
points in the images seen from different angles. In these papers they used the
ultrasonic sensor and 3D vision sensor that detects points of interest on the
bodies of the parked cars and tracking them compare their occurancy positions
from different points of view. The cars are then modeled by vertical planes of
two different orientations that are fit into the 3D data from the sensors. Empty
parking lots are then modeled by empty spaces between the planes.
\newline
The authors of this paper \cite{ichihashi} have developed a system that uses a
system of overhead cameras for occupancy detection in indoor and outdoor scenes.
The focus is put on the clustering algorithms, showing that the performance of
the detector based on the fuzzy c-means (FCM) clustering and the hyperparameter
tuning by particle swarm optimization (PSO) significantly outperforms SVM both
in speed and accuracy of detection.
\newline
The authors of this paper \cite{vladimircoric} move focus from static overhead
cameras on the developed parking lot to estimating parking lots for on-street
parking. In this paper they use an ultra-sonic sensor mounted to the side of the
car to estimate the parking possibilities along particular streets. Uses a
straightforward threshold method to distinguish between free and occupied space.
They also make use of an idea that the more time a car was observed in a place
the more it is likely that the space occupied by this car is a valid parking
spot. As a result all measurements were incorporated with the GPS measurements
to form a global map of parked cars, which the authors used to detect wrongly
parked cars. 
\newline
This paper \cite{schmid11} utilizes an on-board short-range radar. The
measurements from three radars are stored into a 3D occupancy grid, that
represents the local surroundings of the car. Free parking lots are then
detected on the given grid by analyzing cells on curb and on other parked cars.
This provides relatively precise estimation of the parking lot in 3D.
\newline
The authors of this paper \cite{suhr10} proposed a system that is able to detect
parking lots from 3D data acquired from stereo-based multiview 3D
reconstruction. The authors select point correspondences using a
de-rotation-based method and mosaic 3D structures by estimating similarity
transformation. While relying solely on this information and not using the
odometry they were able to achive reliable results with the detection rate of
90\%
\newline
The authors of this paper \cite{suhr13} proposed a system to a fully automatical
detection of parking slots markings. They are utilizing a tree structure
performing a bottom-up and a top-down approach in order to classify all parking
slots as such and leave out all false detections.
