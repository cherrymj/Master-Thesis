%!TEX root = Thesis.tex
\chapter{Related Work}
\label{cha:related_works}

The problem of finding free parking spots receives more and more attention
with the continuous growth of the cities and their population and different
approaches have been proposed to support drivers. We can divide the works in
the field in two main categories depending on the sensor setup. The detection
of the parking spaces is either carried out via the usage of static-mounted
cameras or the on-vehicle sensor systems. We will further focus more on each
of these options.

\section*{Detection with Static Monocular Cameras} % (fold)
\label{sec:detection_with_monocular_cameras}

The main purpose of such systems is to simplify the sensor setup for occupancy
detection in the modern parking garages. Usually the parking lots' occupancy
status is inferred from sensors, placed below or above every parking lot. It
therefore comes in handy to replace all these individual sensors with one
camera, that observes all parking slots at once. The occupancy status of each
parking slot is then inferred from the visual information. These works differ
in the features utilized for car detection as well as in methods of clustering
these features.

\citet{qizhang06} present an unsupervised system to monitor the occupied
parking slots. The authors use a stationary monocular camera to detect parked
vehicles. In their work, car detection is solely based on color. The authors
use support vector machine (SVM) to cluster the features to distinguish
between occupied and free parking slots. They pre-process the ground truth
images of the parking spaces and search for color differences between the
measurements and the ground truth measured for an empty parking lot.

\citet{nicolastrue} proposes a similar approach. Alike with the previous
papers he presents an unsupervised system for parking space detection. The
author utilizes a static overhead camera. He uses human-labeled parking slots'
positions to define the spacial arrangement of the parking spaces. The author
distributes chrominance channels of the images from the camera into bins,
storing a histogram of these values for each parking slot, then he classifies
these histograms as free or occupied using either k-nearest neighbors or
SVM\@. He also presents a variant of the work based on classifying color
patches situated around corner detector's regions of interest, but he proves
it to be less efficient.

\citet{tschentscher} utilize a static monocular camera and show a comparison
of different features for occupancy analysis. As in the works mentioned above,
the authors propose using an overhead camera pointing to the parking lot,
where the distinct parking spaces are manually labeled in the images. They
study the influence of the visual features such as color histograms, gradient
histograms, difference of Gaussian histograms and Haar features on the
detection performance. They also explore the variance in methods for training
a classifier based on the chosen feature set, such as k-nearest neighbors,
linear discriminant analysis and SVM\@. They achieve detection rate of 98\%
while performing in real time.

\citet{ichihashi} shows a system that uses a set of overhead cameras for
occupancy detection in indoor and outdoor scenes. The authors set the focus
towards the clustering algorithms, showing that the performance of the
detector based on the fuzzy c-means (FCM) clustering and the hyper-parameter
tuning by particle swarm optimization (PSO) outperforms SVM both in speed and
accuracy of detection.

\citet{chingchun10} and~\citet{chingjao10}, use a static monocular overhead
camera and a 3-layer Bayesian hierarchical framework (BHF). The authors of the
paper specifically address the challenges of vacant parking space inference
that come from dramatic luminance variations, shadow effect, perspective
distortion, and the inter-occlusion among vehicles showing great detection
rates of up to 99\%.

Some works change the focus from detecting the parked cars to detecting the
appearance of the markers pre-painted on each parking slot. \citet{yusnita12}
proposes a system to detect occupancy status of the parking spaces in parking
garages. They are using an overhead, strictly vertically oriented camera. The
parking spaces are manually labeled by a marker that states their occupancy
state. For each query parking slot the authors produce a binary image which is
then analyzed to find if the parking space is empty or occupied. This process
is carried out by comparing the shape of the detection with the shape of the
prior that reassembles the circle shape of the marker drawn on the floor of
the parking slot.
\newline
\newline

% section detection_with_monocular_cameras (end)

\section*{In-vehicle Detection} % (fold)
\label{sec:in_vehicle_detection}

The main area of interest for us is the in-vehicle detection of the parking
slots. We are specifically interested in the works that present approaches
relying on the fusion of range finder data with visual information.

\citet{fintyelvestri} and~\citet{abadvestri} model free parking slots locally
based on the 3D sonar sensor mounted on the car with conjunction with a visual
3D estimation setup based on the tracking of the interest points in the images
seen from different angles.

In these papers the authors use the ultrasonic sensor and 3D vision sensor
that detects points of interest on the bodies of the parked cars and tracks
them to compare their positions from different points of view. The cars are
then modeled by vertical planes of two different orientations that are fit
into the 3D data from the sensors. The authors model the empty parking slots
by the empty spaces between these planes.

\citet{vladimircoric} also move focus from static overhead cameras on a
developed parking lot to estimating parking slots' positions for on-street
parking. They use an ultra-sonic sensor mounted on the side of the car to
estimate the parking possibilities along particular streets. The authors use a
straightforward threshold method to distinguish between free and occupied
space. They also make use of an idea that the more times a car is observed in
any arbitrary place the more likely it is that the space occupied by this car
is a valid parking slot. As a result, all measurements are incorporated with
the GPS measurements to form a global map of parked cars, which the authors
use to detect wrongly parked cars.

Following the focus on the in-vehicle sensor setup,~\citet{schmid11} utilizes
an on-board short-range radar. The measurements from three radars are stored
into a 3D occupancy grid, that represents the local surroundings of the car.
Free parking spaces are detected on the given grid by analyzing occupancy grid
cells corresponding to curb and other parked cars. This provides an estimate
of the parking space in 3D that allows for precise parking.

\citet{suhr10} proposes a system that is able to detect parking spaces from 3D
data acquired from stereo-based multi-view 3D reconstruction. The authors
select point correspondences using a de-rotation-based method and mosaic 3D
structures by estimating similarity transformation. While relying solely on
this information and not using the odometry they are able to achieve reliable
results with the detection rate of 90\%.

Some works, however focus not on car detection, but on detecting the parking
slots' markings. This approach proves to be useful when searching for free
outdoor parking spaces.

\citet{suhr13} present a fully automatic system for detection of parking slots'
markings. They utilize a tree structure performing a bottom-up and a top-down
approach in order to classify parking slots as such and leave out false
detections.

In this thesis, we present an in-vehicle detection system that is capable of
detecting parked cars, modeling their spacial position as well as occupancy
information inferred from multiple observations. For this task we utilize the
in-vehicle sensor setup, consisting of a stereo camera and, optionally, a
laser range finder. We estimate the occupancy probability for each parking slot
in a global map and present a planner capable of finding a route to a free
parking space considering spacial and occupancy constraints.

% section in_vehicle_detection (end)
