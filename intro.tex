%!TEX root = Thesis.tex
\chapter{Introduction} % (fold)
\label{cha:introduction}

We all have been in a situation, when one was driving around a city,
trying to find a free parking space. One can drive numerous circles around
the center of the city with absolute uncertainty about where and when he
can eventually find a free parking lot.

In Chapter 6 of his work --- ``The geography of transport systems'',
\citet{rodrigue2013geography} argues, that parking in the city centers of
modern big cities with population bigger than one million inhabitants is
one of the most prevalent transport problems. He claims that ``cruising''
in the search for a free curb parking space may account for more then 10\%
of the local circulations as the drivers can spend up to 20 minutes
looking for a parking spot.

There are systems integrated into parking garages, that make help people
to find a free parking space by offering the number of free parking lots
available at the moment. The modern maps usually contain the positions of
such parking garages mapped with their GPS coordinates. These allows these
parking garages to be easily found with an ordinary GPS navigation device
that one can nowadays find in most cell phones. Even though these
occupancy counters make it easier to find a free parking spot, they only
provide the current occupancy information and not its prediction.

Furthermore, it is not always convenient to go to a parking garage as
sometimes they lie far from the wanted destination. In such cases it is a
lot more convenient to leave the car in one of the curb parking spaces.

\citet{shoup2006cruising}, while addressing the problem of adjusting parking
prices in order to reduce time spent on the search for a free parking space,
found that it took between 3.5 and 14 min to find a curb space, and that
between 8 and 74 percent of the traffic was cruising for parking.

This results in having to spend time and fuel on repeatedly driving along the
same route in a desperate hope that someone has left and there is a parking
spot available now. There is significant amount of uncertainty in this task.
And uncertainty is the thing that may lead to frustration.

\citet{rodrigue2013geography} evaluates the parking difficulty by asking the
parkers to express their parking impressions. His results indicate that the
amount of parking information parkers had before their trips was directly
related to their parking search time, which in turn, influenced their
perceptions of parking difficulty.

Dealing with such uncertainty is annoying for humans. Yet, people deal with
such uncertainty much better then the robotized systems. In order to carry out
the parking task in an autonomous fashion there needs to be more information
provided. There must be a deterministic algorithm, that has a decision what an
optimal action is in any moment in time. For this algorithm to function there
has to be enough information on spacial representation of the parking spaces
as well as a model, that accounts for the occupancy information.

To the extent of our knowledge, there is, only a few solutions that provide
information on the positions and occupancy of the curb parking spaces.

One one the best known examples is the ``Parking on demand'' system
by~\citet{sfo,sfo2}. They present a system of parking meters that adapt the
price of curb parking spaces with relation to current occupancy in the area.
Along with intelligent parking meters they also present a smartphone
application that provides live information on the occupancy and pricing of
distinct parking spaces.

Even though this system proves to save time and money for the people that use
it, we still see a lack of automatism in it. In order for this system to be
used in an autonomous fashion one has to introduce a planner that considers
not only spacial information but also occupancy and pricing.

Another down side of this system is its dependence on the fixed positions of
the parking meters. This implies, that in order to use the presented system we
first need to install the parking meters for every curb parking space. This
can be an effort, time and money consuming task.

We argue, that there has to be a system to be able to map the paring spaces,
estimate their occupancy information and search for one in an optimal and
fully autonomous way.

Autonomous vehicles are becoming a topic of a great interest. In the latest
years, automotive and software companies as well as the universities all over
the world have focused on developing their own autonomous vehicles. These
vehicles can already impressively well navigate in the
cities~\cite{stanley_auto_car,perceprion_drivec_car,lima13,daimler} and are able to
take people to an arbitrary point even in difficult urban environments. They
can already learn how to adopt behavioral knowledge from the
traffic~\cite{behaviour_learning,spinello10:multiclass}. We believe, that
these cars are to become safer, more efficient in the means of fuel
consumption and more predictable than the human drivers. Following the recent
success of the Google self driving car~\cite{markoff2010google} and the fact
that it is now officially allowed to use self-driving autos on public roads in
California, Nevada and Florida, US, we predict that in the upcoming years,
people will spend less time driving by themselves and will rely more on
robotized autos.

When it comes to parking, these vehicles can find out what a parking space is
and are able to park
there~\cite{auto_cars_burgard,auto_parking09,auto_park2_11}. Despite being
able to park in an autonomous fashion, they are yet unable to know where to
search for a free parking slot by themselves and, as a consequence, where to
search for a good one.

We address precisely this issue. We present an approach that provides a system
for mapping the parking spaces with estimating their occupancy probability via
integrating information from multiple observations. Not only we are able to
estimate this information but we also present a way to plan the route in such
way that it eliminates uncertainty in finding the free parking place and
guarantees to minimize the time spent on cruising while searching for a free
parking space and walking from the found one to the destination.

% chapter introduction (end)
